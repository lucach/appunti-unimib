\chapter{Seconda lezione (09/10/2015)}

\section{Estremi e limiti di insiemi}

\begin{definition}
Un sottoinsieme $A \subseteq \Q$ è limitato superiormente se esiste $k \in \Q$ tale che $k \ge x$ per ogni $x \in A$. Tale $k$ è detto maggiorante di $A$. 

In modo analogo, $A$ è limitato inferiormente se esiste $k \in \Q$ tale che $k \le x$ per ogni $x \in A$. Tale $k$ è detto minorante di $A$.
\end{definition}

Dati $a \neq 0$, l'estremo superiore ($\sup A$), il minimo dei maggiornati, l'estremo inferiore ($\inf A$) e il massimo dei maggioranti (purché esistano):
\begin{itemize}
\item se $A$ non è limitato superiormente $ \implies \sup A = + \infty$
\item se $A$ non è limitato inferiormente $ \implies \inf A = - \infty$
\end{itemize}

\begin{remark}
Se $A$ ha un massimo $x$, allora $x \in \sup A$.

Infatti, essendo un massimo, $x$ è un maggiorante e vale $x \le y$ perché $x \in A$ è il minimo dei maggioranti.
\end{remark}

Ad esempio, dato $A = \set{x \in \Q | 0 < x < 1}$, è evidente che $\sup A = 1$ ma $1 \notin A$. In questo caso l'insieme $A$ non ha massimo.

\begin{example}
Dato $A = \Q$ osserviamo che non è limitato superiormente. Questo perché $x \in \Q$ è un maggiorante se $x \ge y$ per ogni $y \in \Q$; dovrebbe essere quindi $x \ge x + 1$ che è assurdo. Quindi $\sup \Q = +\infty$.
\end{example}

\begin{example}
Dato $A = \Q$ non vuoto, se $A$ è limitato superiormente allora $A$ ha un massimo.

Infatti $S = \set{x \in \Z | x \text{ è maggiorante di } A}$. $S$ non può essere vuoto perché $A$ è limitato superiormente.

Allora $S$ ha un minimo $x$ per il principio del buon ordinamento. Essendo il minimo, $x-1$ non è un maggiorante.

Quindi esiste un $y \in A$ tale che $y > x-1$ e $y \le x$. L'unico caso possibile è che $x$ e $y$ coincidano, ovvero $y = x \in A$

Quindi $x$ è un maggiorante e appartiene ad $A$. Quindi $x$ è il massimo.
\end{example}

\begin{proposition}
Dato $A \in \Q$ non vuoto e $y \in \Q$, $y$ è l'estremo superiore di $A$ se e solo se $y$ è maggiorante di $A$ e per ogni $y' < y$ esiste $x \in A$ tale che $y' < x \le y$.
\end{proposition}

Osserviamo che la condizione $x \le y$ è una diretta conseguenza del fatto che $y$ è un maggiorante.

Dimostriamo la proposizione in entrambi i sensi, per mostrare che vale l'implicazione ``se e solo se''.

\begin{proof}
Se $y$ è estremo superiore, allora devo dimostrare che:
\begin{enumerate}[I.]
\item $y$ è un maggiorante
\item $\forall y' < y \quad \exists x \in A \quad y' < x \le y$
\end{enumerate}
Il punto I è ovvio; dimostriamo il punto II.

Dato $y' < y$, $y'$ non può essere maggiorante perché $y$ è il minimo dei maggioranti. Quindi esiste $x \in A$ tale che $y' < x$. Poiché $y$ è un maggiorante, possiamo scrivere $y' < x \le y$.
\end{proof}

Dimostriamo ora il viceversa.

\begin{proof}
Dobbiamo dimostriare che i punti I e II implicano il fatto che $y$ sia un estremo superiore.

Sia $y'$ maggiorante di $A$. Se $y' < y$ allora esiste $x \in A$ tale che $y' < x \le y$, quindi $y'$ non è un maggiorante; il che è assurdo.

Allora ogni maggiorante $y'$ deve essere $y' \ge y$. Poiché $y$ è un maggiorante, $y$ è il minimo dei maggioranti. Quindi $y = \sup A$.
\end{proof}

\begin{example}
Proviamo a calcolare $\sup A$ di:
\begin{equation*}
A = \left\{\frac{n-1}{n+1} \; \mid \; n \in \N \right\}
\end{equation*}
Che è l'insieme:
\begin{equation*}
A = \left\{0, \frac{1}{3}, \frac{1}{2}, \frac{3}{5}, \ldots \right\}
\end{equation*}
Ha senso innanzitutto chiedersi se $A$ è limitato superiormente. Possiamo dire che lo è con certezza perché il numeratore è sempre inferiore al denominatore, quindi
\begin{equation*}
\frac{n-1}{n+1} < 1
\end{equation*}
Quindi $1$ è un maggiorante. Dimostriamo che è anche il $\sup$ ($\sup A = 1$).

In altri termini dobbiamo dimostrare che preso $y' < 1$ esiste $x \in A$ tale che $y' < x \le 1$, che equivale a risolvere:
\begin{equation*}
y' < \frac{n-1}{n+1}
\end{equation*}
\begin{equation*}
y' (n+1) < n-1
\end{equation*}
\begin{equation*}
y'n + y' < n -1
\end{equation*}
\begin{equation*}
n(1-y')>y'+1
\end{equation*}
\begin{equation*}
n > \frac{y'+1}{1-y'}
\end{equation*}
Poiché $\N$ non è limitato superiormente, esiste sempre una soluzione $n \in N$. Ovvero:
\begin{equation*}
\exists x = \frac{n-1}{n+1} \in A \qquad y' < x
\end{equation*}
Quindi $\sup A = 1$.
\end{example}

\section{Insieme $\R$}

I numeri reali sono un insieme, chiamato $\R$, su cui sono definite le operazioni di somma e prodotto, è definito un ordinamento ed esistono due elementi neutri per le due operazioni precedenti.

L'insieme dei numeri reali soddisfa tutte le seguenti proprietà, che erano già soddisfatte da $\Q$ ma che riportiamo:
\begin{itemize}
\item La somma è commutativa e associativa, ha un elemento neutro che è lo zero.
\item Il prodotto è commutativo e associativo, ha un elemento neutro che è l'uno.
\item Di ogni elemento esiste l'opposto e l'inverso, ovvero:
\begin{equation*}
\forall a \in \R \text{ esiste } b \in \R \qquad a + b = 0 \quad (b=-a)
\end{equation*}
\begin{equation*}
\forall a \in \R\backslash\{0\} \text{ esiste } b \in \R \qquad a\cdot b = 1 \quad \left(b=\frac{1}{a}\right)
\end{equation*}
\item È definito un ordinamento:
\begin{equation*}
a \le b \text{ e } b \le a \implies a = b
\end{equation*}
\begin{equation*}
a \le b \text{ e } b \le c \implies a \le c
\end{equation*}
\begin{equation*}
\forall a, b \quad a \le b \text{ o } b \ge a
\end{equation*}
\begin{equation*}
a \le b \implies \forall c \quad a + c \le b + c
\end{equation*}
\begin{equation*}
a \le b \implies \forall c > 0 \quad a \cdot c \le b \cdot c
\end{equation*}
\end{itemize}

In aggiunta alle proprietà comuni a $\Q$ abbiamo che se $A \subseteq \R$ è non vuoto, allora $A$ ammette un estremo superiore (eventualmente $\sup  A = +\infty$).

Dobbiamo mostrare ora che esistono i numeri reali e per fare ciò ricorriamo al modello degli \emph{allineamenti decimali}.

Un \emph{allineamento decimale} è una sequenza numerabile di interi del tipo $p_0, p_1, p_2, \ldots $ dove
\begin{equation*}
0 \le p_k \le 9 \qquad k \in \N
\end{equation*}

Un allineamento decimale è \emph{periodico} se si ripete da un certo punto in poi, cioè ha la forma:
\begin{equation*}
p_0, p_1, \ldots, \underbrace{p_n, p_{n+1}, \ldots, p_{n+k}}_{\text{periodo}}, p_{n+1+k}, \ldots
\end{equation*}

Un allineamento periodico con periodo 0 si dice anche che è limitato (ad esempio $1,5000...$ è limitato).

Dato un allineamento decimale $x$ definiamo il troncamento $k$-esimo
\begin{equation*}
r_k(x) = p_0 + \frac{1}{10}p_1 + \ldots + \frac{1}{10^k} p_k
\end{equation*}

Il troncamento $r_0(x)$ è la \emph{parte intera} di $x$ e si indica $[x]$.

Ad ogni $x \in \Q$ possiamo associare un allineamento decimale $T(x) = p_0, p_1, p_2, \ldots$.
\begin{equation*}
p_0 = \max \set{n \in \Z | n \le x}
\end{equation*}
\begin{equation*}
p_1 = \max \set{n \in \Z | p_0 + \frac{1}{10}n \le x}
\end{equation*}
e così via, in modo che per ogni $k > 0$ sia
\begin{equation*}
p_k = \max \set{n \in Z | \underbrace{r_{k-1}(T(x))}_{p_0+\frac{1}{10}p_1 + \ldots} + \frac{1}{10^k} n \le x}
\end{equation*}

\begin{example}
Consideriamo $T(x)$ per $x = \frac{3}{2}$, che è $1,5000...$.
Infatti:
\begin{equation*}
p_0 = \max \set{n \in \Z | n \le \frac{3}{2}} = 1
\end{equation*}
\begin{equation*}
p_1 = \max \set{n \in \Z | 1 + \frac{1}{10}n \le \frac{3}{2}} = 5
\end{equation*}
\begin{equation*}
p_2 = \max \set{n \in \Z | 1 + \frac{5}{10}n + \frac{1}{100}n \le \frac{3}{2} } = 0
\end{equation*}

Prestiamo attenzione al fatto che il comportamento di $T(x)$ per i numeri negativi, così per come è stato definito, è diverso da come ce lo potremmo aspettare. Ad esempio:
\begin{equation*}
T(-\frac{3}{2}) = -2,500...
\end{equation*}
\end{example}

\begin{proposition}
Per ogni $x \in \Q$, $k > 0$ vale:
\begin{equation*}
0 \le x - r_k(T(x)) < \frac{1}{10^k}
\end{equation*}
Come conseguenza si ha che l'insieme
\begin{equation*}
\set{n \in \Z | r_{k-1}((T(x)) + \frac{1}{10^k}n \le x}
\end{equation*}
contiene 0 e ha 9 come maggiorante.
\end{proposition}