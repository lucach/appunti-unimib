\chapter{Prima lezione (06/10/2015)}

\section{Insieme $\N$}

\begin{definition}
L'insieme $\N$ è l'insieme dei numeri interi positivi, detti numeri naturali, e si indica con $\N=\{1,2,3,...\}$.
\end{definition}

Su di esso sono definite due operazioni:
\begin{itemize}
\item Somma: $\N + \N \to \N$, quindi $(a, b) \to a + b$
\item Prodotto: $\N \cdot \N \to \N$, quindi $(a, b) \to a\cdot b$
\end{itemize}

Queste due proprietà sono commutative e associative:
\begin{itemize}
\item $a+b=b+a$
\item $a+(b+c)=(a+b)+c$
\item $a\cdot b = b\cdot a$
\item $a\cdot(b\cdot c) = (a\cdot b)\cdot c$
\end{itemize}

Vale inoltre la proprietà distributiva:
\begin{equation*}
(a+b)\cdot c = a\cdot c + b\cdot c
\end{equation*}

Nel prodotto esiste un elemento neutro, in altri termini esiste un $e \in \N$ tale per cui, comunque scelto $a$, $a \cdot e = e \cdot a = a$. Tale $e$ risulta ovvio essere 1.

Nell'insieme $\N$ esiste una relazione di ordinamento ($a \le b$) tale per cui:
\begin{enumerate}[I.]
\item $a \le b$ e $b \le a \implies a = b$
\item $a \le b \le c \implies a \le c$
\item $\forall a,b \;\; a \le b$ oppure $b \le a$
\end{enumerate}


\begin{definition}
Un insieme S con una relazione d'ordine che soddisfa I, II, III si dice totalmente ordinato.
\end{definition}

\begin{remark}
Ogni $S \subseteq \N$ è totalmente ordinato.
\end{remark}

Se $a \le b$ e $c \in \N \implies a+c \le b+c$

Se $a \le b$ e $c \in \N \implies a \cdot c \le b \cdot c$ 

L'equazione $n+x=m$ ha una soluzione (unica) se e solo se $m > n$.

Anche $\set{x \in \Q | x > 0}$, l'insieme dei numeri razionali, soddisfa le condizioni sopra indicate.

\begin{definition}
Dato un insieme totalmente ordinato (scriviamo $(S, \le)$), $X$ è il minimo di $S$ se $x \in S$ e per ogni $y \in S$ vale $x \le y$.
\end{definition}

\begin{proposition}[\bfseries Principio del buon ordinamento]
Ogni sottoinsieme di $\N$ non vuoto ha un minimo.
\end{proposition}

\begin{example}
L'inisieme $\set{x \in Q | x > 0}$ non soddisfa il principio del buon ordinamento perché, ad esempio, il suo sottoinsieme $\set{\frac{1}{n} | n > 0}$ non ha minimo.
\end{example}

\begin{remark}
Grazie al principio del buon ordinamento vale: 
\begin{equation*}
\set{x \in \N | x \le s} = \{1, \ldots, s\}
\end{equation*}
\end{remark}

\begin{proposition}[\bfseries Principio di induzione]
Sia $P_n$ un enunciato che dipende da $n \in \N$ (ad esempio ``$n$ è pari'', ``$n$ è primo''). Supponiamo che $P_1$ sia vero e che valga l'implicazione $P_n \implies P_{n+1}$. Allora $P_n$ è vero per ogni $n$.
\end{proposition}

\begin{proof}
Sia $S = \set{n \in \N | P_n \text{ è falso}}$.
Se $S = \emptyset$ non c'è niente da dimostrare.
Altrimenti, per il principio del buon ordinamento $S$ ha un minimo $k = min\; S$. Non può essere $k = 1$ ($1 \in S$) perché $P_1$ è vero.

Essendo $k>1$, $k-1 \in \N$ (ricorda l'equazione $1+k=x$) e $k-1 \in S$.

Allora $P_{k-1}$ non è falso, quindi $P_{k-1}$ è vero. $P_k$ è vero per ipotesi. Ma questo contraddice l'ipotesi che $k \in S$, quindi il caso $S$ non vuoto non si verifica.
\end{proof}

Nota che, ad esempio, l'enunciato ``$\forall n,\;n > 0$'' non è un enunciato che dipende da $n$!

\begin{example}
Dimostriamo per induzione che
\begin{equation*}
P_n : \sum\limits_{i=1}^n i = \frac{1}{2} \cdot (n+1) \cdot n
\end{equation*}

Verifichiamo $P_1$: 
\begin{equation*}
P_1 : \sum\limits_{i=1}^1 i = \frac{1}{2} \cdot (1+1) \cdot 1
\end{equation*}
che equivale a $1 = 1$ ed è quindi vero.

Ora dobbiamo verificare anche che $P_n \implies P_{n+1}$.

\begin{equation*}
P_n : \sum\limits_{i=1}^n i = \frac{1}{2} \cdot (n+1) \cdot n
\end{equation*}
\begin{equation*}
P_{n+1} : \sum\limits_{i=1}^{n+1} i = \frac{1}{2} \cdot (n+2) \cdot (n+1)
\end{equation*}

Per definizione vale anche che :

\begin{equation*}
P_{n+1} : \sum\limits_{i=1}^{n+1} i = \sum\limits_{i=1}^n i + (n + 1) =  \frac{1}{2} \cdot (n+1) \cdot n + (n+1)
\end{equation*}

\begin{equation*}
\sum\limits_{i=1}^{n+1} i = \frac{1}{2} (n+1) (n+2)
\end{equation*}

\end{example}

\section{Insieme $\Z$}

Consideriamo queste due equazioni:
\begin{itemize}
\item $a+x=b$, che ha soluzione in $\N$ se e solo se $b > a$.
\item $a \cdot x = b$, che ha soluzione in $\N$ quando $a$ è un divisore di $b$ (si scrive $x = \frac{b}{a}$).
\end{itemize}

È evidente che serve quindi estendere l'insieme $\N$ arrivando all'insieme degli interi $\Z$ così definito:

\begin{equation*}
\Z = \{\dots, -1, 0, 1, \dots \}
\end{equation*}

$\Z$ è la più piccola estensione di $\N$ dove l'equazione $a+x=b$ ha soluzione per ogni $a, b$. 

In $\Z$ valgono le stesse proprietà di $\N$.

$\Z$ ha un elemento neutro per la somma (zero). Ovvero scriviamo:
\begin{equation*}
a + 0 = 0 + a = a \;\;\; \forall a
\end{equation*}

Dato $a \in \Z$ esiste $x \in \Z$ tale che $a+x=0$ (si scrive $x=-a$).

Per passi:
\begin{align*}
b - a = b + (-a) \\
a + (b-a) = b
\end{align*}

che è la soluzione di $a+x=b$ cercata.

Nota inoltre che $a \cdot x = b$ non ha soluzioni per $a=0,\;b\neq 0$ perché $0 \cdot x = 0$, che a sua volta discende da
\begin{align*}
1 \cdot x &= (1 + 0) \cdot x \\
&=1 \cdot x + 0 \cdot x
\end{align*}

Sottraendo $(1 \cdot x)$ a entrambi i membri risulta $0 = 0 \cdot x$.

\section{Insieme $\Q$ e oltre}

Definiamo l'insieme $\Q$, insieme dei numeri razionali, in questo modo:
\begin{equation*}
\Q = \set{\frac{p}{q} | p,q \in \Z,\;q \neq 0}
\end{equation*}

$\Q$ ha le stesse proprietà di $\Z$. Inoltre:
\begin{equation*}
\forall a \neq 0 \;\; \exists\, x \in \Q : a \cdot x = 1
\end{equation*}

$x = \frac{1}{a}$, da cui $\frac{b}{a} = b \cdot \frac{1}{a}$ che è la soluzione di $a \cdot x = b$. 

Infatti:

\begin{equation*}
a \cdot \frac{b}{a} = a \cdot b \cdot \frac{1}{a} = b \left(a\left(\frac{1}{a}\right)\right) = b \cdot 1 = b
\end{equation*}

È evidente che i numeri razionali non vanno bene per l'analisi numerica. Supponiamo di voler misurare un segmento in gessetti: potrebbero volerci quattro gessi ``e un pezzetto''. Potremmo dividere il gessetto a metà e scoprire che la lunghezza del segmento è 4 gessi + 1 gessetto + ``un pezzettino''. Non è detto che questo processo termini! Infatti non tutti gli intervalli si possono rappresentare con un numero razionale.  

\begin{proof}
Sia $x$ la diagonale di un quadrato di lato 1. Per Pitagora vale che $x^2 = 1 + 1 = 2$. Se $x$ fosse razionale, potremmo scrivere $x = \frac{p}{q}$ per un qualche $p,q \in \Z$. 

Quindi varrebbe $\frac{p^2}{q^2} = 2$, ovvero $p^2 = 2\cdot q^2$.

Possiamo scrivere $p = 2^k \cdot a$ per un qualche $a$ dispari e $q = 2^h \cdot b$ per un qualche $b$ dispari.

Sostituendo nella prima equazione resta: $2^{2k} \cdot a^2 = 2 \cdot 2^{2h} \cdot b^2$.

$a^2$ e $b^2$ sono quadrati di un numero dispari e quindi dispari anch'essi.

Se uguagliamo gli esponenti risulta $2k = 2h + 1$ dove il primo è un numero pari mentre il secondo è un numero dispari, il che è assurdo.

Quindi, $x^2 = 2$ non ha soluzione in $\Q$.
\end{proof}

\section{Estremo superiore e maggioranti}

\begin{definition}
Un sottoinsieme $A \subseteq \Q$ è \emph{limitato superiormente} se esiste un $k \in \Q$ tale che $a \le k$ per ogni $a \in A$. 

Un tale $k$ è detto \emph{maggiorante} di $A$.
\end{definition}

\begin{definition}
Dato $A \subseteq \Q$ non vuoto e limitato superiormente, si dice \emph{estremo superiore} di $A$ il minimo dei maggioranti, se esiste. (Si indica $\sup A$.)
\end{definition}

Se $A$ è non vuoto ma non è limitato superiormente, allora $\sup A = + \infty$.

\begin{example}
Sia $A = \set{x \in \Q | 0 < x < 1}$.
Esso è limitato superiormente perché se prendo $k = 2$, $k > a \;\; \forall a \in A$.

$y$ è maggiornate di $A \implies y > x \; \; \forall x \in A$.

Sia $y \in \Q$:
\begin{itemize}
\item Se $y \ge 1$ allora $y$ è un maggiorante.
\item Se $0 < y < 1$, supponiamo $x = \frac{1}{2}(y+1)$ (ovvero $x$ punto medio tra $y$ e $1$). Vale che $0 < x < 1 \implies x \in A$. Poiché $x > y$, $y$ non è un maggiornate.
\item Se $y < 0$ supponiamo $x = \frac{1}{2} \in A$; $x > y$ quindi $y$ non è un maggiorante.
\end{itemize}

In definitiva i maggioranti sono $\set{y \in \Q | y \ge 1}$ e $\sup A = 1$.
\end{example}

\begin{example}
Sia $A = \set{x \in \Q | x^2 \le 2}$. $A$ è limitato superiormente.\end{example}

\begin{proposition}
$2$ è maggiorante di $A$.
\end{proposition}
\begin{proof}
Supponiamo che $2$ non sia maggiorante. Allora non è vero che $x \le 2 \; \; \forall x \in A$. Quindi esiste $x \in A$ tale che $x > 2$. Allora $x^2 > 2^2$, ovvero $x^2 > 4$ che è assurdo perché vale che $x^2 < 2$.
\end{proof}
\begin{proposition}
$A$ non ha un estremo superiore in $\Q$.
\end{proposition}
\begin{proof}
Sia $x \in \Q$ un maggiorante. Allora $x^2 \neq 2$. 

\begin{itemize}
\item 
Se $x^2 < 2$, consideriamo l'insieme $(x+\frac{1}{n})^2$ al variare di $n$ (sviluppando: $x^2 + \frac{2}{n} + \frac{1}{n^2}$).
Per $n$ sufficientemente grande $y = x + \frac{1}{n}$. Essendo $y^2 < 2$, basta che $\frac{2}{n} + \frac{1}{n^2} \le 2 - x^2$.

Ovvero
\begin{equation*}
(2 - x^2) \cdot n^2 - 2n + 1 > 0 
\end{equation*} 

Nota che l'equazione sopra è una parabola con concavità verso l'alto. 

Allora $x$ non è un maggiorante perché $x < y$ e $y \in A$.

\item 
Se $x^2 > 2$ allora $y = x - 1$ è maggiorante.

\begin{align*}
(x-\frac{1}{n})^2 &> 2 \\
x^2 - \frac{2}{n} + \frac{1}{n^2} &> 2 \\
n^2 \cdot (x^2 - 2) - 2n + 1 &> 0
\end{align*}

che è vera per $n$ sufficientemente grande.

\end{itemize}

Quanto sopra implica che deve esistere un maggiorante della forma $y = x - \frac{1}{n}$. Ciò implica che $x$ non è il minimo dei maggioranti e a sua volta questo implica che $A$ non ha $\sup$.

\end{proof}

