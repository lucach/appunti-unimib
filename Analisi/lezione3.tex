\chapter{Terza lezione (13/10/2015)}

\section{Allineamenti decimali e insieme $\R$}

Abbiamo già visto come si scrive un allineamento decimale: $P_0, P_1, P_2, \dots$ con $P_k \in \Z$, $0 \le P_k \le 9$ per $k > 0$.

Dato un allineamento $x$ consideriamo il suo $k$-esimo troncamento: 
\begin{equation*}
r_k(x) = P_0 + \frac{1}{10}P_1 + \ldots + \frac{1}{10^k}P_k
\end{equation*}

Dato $x \in \Q$, esiste l'allineamento decimale $T(x)$ tale che $0 \le x - r_k(T(x)) \le \frac{1}{10^k}$.

Nota che $T(x)$ non può avere periodo 9.

\begin{example}
Supponiamo che esista $T(x) = 0,\overline{9}$. Allora
\begin{align*}
r_k(T(x)) &= 0 + \frac{9}{10} + \ldots + \frac{9}{10^k} \\
&= \frac{10^k-1}{10^k}
\end{align*}

Ad esempio per $k = 2$ varrebbe $r_k(T(x)) = \frac{9}{10} + \frac{9}{100} = \frac{99}{100}$; e così via.

\begin{equation*}
0 \le x - r_k(T(x)) < \frac{1}{10^k}
\end{equation*}

Che è equivalente a:

\begin{equation*}
\underbrace{r_k(T(x))}_{\frac{10^k-1}{10^k}} \le x < \underbrace{r_k(T(x)) + \frac{1}{10^k}}_{1}
\end{equation*}

Non esiste $x \in \Q$ tale che $1 - \frac{1}{10^k} \le x < 1$ per ogni $k$. Ciò implica che $0,\overline{9}$ non è $T(x)$ per un $x \in \Q$.

\end{example}

\begin{definition}
Un allineamento decimale è ammissibile se non è periodico con periodo 9.
\end{definition}

Sia $\mathcal{A}$ l'insieme degli allineamenti decimali ammissibili. Definiamo $T$ come la funzione che associa un numero razionale a un allineamento ammissibile (che è un elemento dell'insieme $\mathcal{A}$). Sinteticamente si scrive:

\begin{equation*}
T : \Q \rightarrow \mathcal{A}
\end{equation*}

Poniamo $\R = \mathcal{A}$. L'ordinamento su $\mathcal{A}$ è definito nel seguente modo: $p_0, p_1, \dots, p_k < q_0, q_1, \dots, q_k$ se e solo se, detto $k = \min \set{i | p_i \neq q_i}$, si ha $p_k < q_k$. 

\begin{example}
Consideriamo il banale ordinamento tra le seguenti coppie di allineamenti:
\begin{itemize}
\item $2,3 < 3,2$ (vera per $k = 0$)
\item $1,12 < 1,13$ (vera per $k = 2$)
\end{itemize}
\end{example}

\begin{definition}
Dati $x, y \in \mathcal{A}$ definiamo $x \le y$ se $x < y$ o $x = y$.
\end{definition}

L'insieme $\mathcal{A}$ è totalmente ordinato.

\begin{proposition}
Ogni $X \subset \mathcal{A}$ non vuoto ha un estremo superiore.
\end{proposition}

\begin{proof}
Se $X$ non è limitato superiormente, allora $\sup X = +\infty$.

Se $X$ è limitato superiormente, allora esiste sicuramente un maggiorante $M$. \\
Per ogni $k \in \Z$ con $k \ge 0$ definiamo la funzione $a_k : \mathcal{A} \rightarrow \Z$ (che estrae la $k$-esima cifra); ovvero: $a_k(p_0, p_1, \dots, p_k) = p_k$.

Osserviamo che $\set{a_0(z) | z \in X}$ è limitato superiormente perché $M$ è un suo maggiorante. Ad esempio se $X = \{1.2, 2.3, 3.4 \}$ allora $\set{a_0(z) | z \in X} = \{1, 2, 3\}$ e quindi $M = 4$.

Sia $q_0 = max \set{a_p(z) | z \in X}$. Considerando ancora l'esempio sopra, in questo caso $q_0 = 3$.

Se $k > 0$: $\set{a_k(z) | z \in X} \subseteq \{0, \dots, 9\}$.

$q_k = max \set{a_k(z) | z \in X \text{ tale che } a_0(z)=q_0, a_1(z)=q_1, \dots, a_{k-1}(z) = q_{k-1}}$.

Sia $y = q_0, q_1, \dots$ un maggiorante.
Sia $z = p_0, \dots, p_k$ di X; sia $j = min \set{p_j \neq q_j}$ con $z \neq y$.
\begin{equation*}
q_j = max \underbrace{\set{a_j(z) | z \in X a_0(z) = q_0, \dots, a_{j-1}(z) = q_{j-1}}}_{C}
\end{equation*}

Notiamo che $C$ contiene $z$. Ciò implica che $\underbrace{a_j(z)}_{p_j} \le q_j$, ma per la definizione precedente $p_j \neq q_j$. 

Ciò implica $p_j < q_j$, che a sua volta implica $z < y$. Quindi $y$ è effettivamente un maggiorante. \\

Preso $y' \le y$ devo dimostrare che $\exists z \in X$ tale che $y' < z$. \\
$y' = p_0, p_1, \dots$ \\
$y = q_0, q_1, \dots$.\\
Supponiamo che $p_i = q_i$ e $p_k < q_k$ per ogni $i < k$. Per definizione di $q_k$ esiste $z \in X$ tale che $a_i(z) = q_i$ per $i \le k$. Per costruzione questo implica $y' < z$. Abbiamo quindi dimostrato che $y = \sup X$.
\end{proof}

Per ora abbiamo definito $(\mathcal{A}, \le)$. Dobbiamo però ancora definire la somma in $\mathcal{A}$. Si pone: $x + y = \sup \set{T(r_k(x)+r_k(y)) | k \in \N}$ dove $x, y$ sono allineamenti. 

Possiamo inoltre definire in modo analogo il prodotto.

Per $x, y \ge 0$, $x \cdot y = \sup \set{T(r_k(x) \cdot r_k(y)) | k \in \N}$

\begin{proposition}
Esiste un $e \in \mathcal{A}$ tale che $x + e = x = e + x$ per ogni $x$ (detto anche ``zero'').
\end{proposition}

\begin{proof}
Poniamo $e = T(0) = \{0,0000\dots\}$. 

Allora $x+e=\sup \set{T(r_k(x)+r_k(e)) | k \in \N}$.

Calcoliamo $r_k(e) = 0 + \frac{1}{10}\cdot 0 + \dots + \frac{1}{10^k}\cdot 0 = 0$.

Quindi $x + e = \sup \set{T(r_k(x)) | k \in \N} = x$.

Quindi $\mathcal{A}$ contiene lo zero.

\end{proof}

In modo del tutto analogo si prova che $\mathcal{A}$ contiene anche $1,000\dots$.

Siamo quindi pronti per definire l'insieme dei reali $\R$; in modo sintetico scriviamo $\R = (\mathcal{A}, \le, +, \cdot, 0, 1)$.

Osserviamo che $\Q \subseteq \R$ e che ogni $x \in \Q$ determina un $T(x) \in \mathcal{A} = \R$.

Valgono le solite proprietà:
\begin{itemize}
\item $T(x+y) = T(x) + T(y)$
\item $T(x \cdot y) = T(x) \cdot T(y)$
\item $T(0) = 0$.
\item $T(1) = 1$.
\end{itemize}

\begin{proposition}[\bfseries Proprietà di Archimede]
Dati $a, b$ reali positivi esiste un $n \in \N$ tale che $n \cdot a > b$.
\end{proposition}

\begin{proof}
Per assurdo supponiamo che valga il contrario, ovvero che $n \cdot a < b \; \; \forall n$. Allora $n < \frac{a}{b}$. Questo è impossibile perché $\N$ dovrebbe essere limitato superiormente, quindi avere un massimo. Ma ciò è palesemente assurdo, perché vale sempre $x + 1 \in N$ e $x + 1 > x$.
\end{proof}

\section{Potenze e logaritmi}

Dato $a \in \R$ e $n \in \N$ definiamo $a^n = \underbrace{a \cdot \ldots \cdot a}_{n\;volte}$. L'elevamento a potenza gode delle seguenti proprietà:
\begin{itemize}
\item $a^{n+m} = a^n \cdot a^m$
\item $(a \cdot b)^n = a^n \cdot b^n$
\item $(a^n)^m = a^{n \cdot m}$
\end{itemize}

Per definizione se $a \neq 0 \implies a^0 = 1$. Sempre per definizione $a^{-n} = (\frac{1}{a})^n$.

\begin{theorem}
Dato $x \in \R$ positivo e $n \in \N$ esiste un unico reale positivo, $y$, tale che $y^n = x$ (ovvero $y = \sqrt[n]{x}$).
\end{theorem}

\begin{definition}
Dato $x$ reale positivo e $\frac{p}{q} \in \Q$ (assumendo senza perdita di generalità $q > 0$), si pone $x^{\frac{p}{q}} = (\sqrt[q]{x})^p$.

Se $x \ge 1$ reale e $y \in \R$ definiamo $x^y = \sup \set{x^\frac{p}{q} | \frac{p}{q} \le y}$.

Se $x < 1$ possiamo invertire: $x^y = (\frac{1}{x})^{-y}$.

Valgono le solite proprietà.
\end{definition}

\begin{theorem}
Dato $x \in \R$ (con $x > 0$, $y > 1$) esiste un unico $z \in \R$ tale che $x^z = y$. Si scrive: $z = \log_x y$.
\end{theorem}

\section{Intervalli e intorni}
Dati $a, b \in \R$ sono definiti i seguenti intervalli (riportati solo nelle forme più esemplificative, le altre sono immediate dalle seguenti):
\begin{align*}
(a,b) &= \set{x \in \R | a < x < b} \\
[a,b] &= \set{x \in \R | a \le x \le b} \\
(-\infty,a) &= \set{x \in \R | x < a} \\
(a, +\infty) &= \set{x \in \R | x > a} \\
[a,b) &= \set{x \in \R | a \le x < b} \\
(a,b] &= \set{x \in \R | a < x \le b}
\end{align*}

\begin{definition}
Dati $x \in \R$ e $r \in \R$ (con $r > 0$), si dice intorno circolare di $x$ di raggio $r$ l'intervallo $B_r(x) = (x-r, x+r) = \set{y \in \R | |x-y| < r}$.


Definiamo inoltre $B_r'(x) = B_r(x)\backslash\{x\} = (x-r, x) \cup (x, x+r)$.
\end{definition}

Ricorda che $|x| = 
\begin{cases} 
x & \mbox{se } x \ge 0 \\
-x & \mbox{se } x < 0 \\ 
\end{cases} $

Inoltre osserviamo che $|x+y| \le |x| + |y|$.

\section{Successioni}

Una \emph{successione} è una funzione $x:\N \rightarrow \R$ ($n \rightarrow x_n$).

Tale funzione viene rappresentata con la notazione $\{x_1, x_2, x_3, \dots\}$ oppure $\{x_n\}_{n \in \N}$.

\begin{example}
$\{n\}_{n \in \N}$ rappresenta la successione $\{1, 2, 3, \dots \}$, ovvero la funzione $\N \rightarrow \N$ ($n \rightarrow n$).
\end{example}

\begin{example}
$\{[\sqrt{n}]\}_{n \in \N}$ rappresenta la successione $\{1, 1, 1, 2, \dots, [\sqrt{n}], \dots \}$, ovvero la funzione $\N \rightarrow \N$ ($n \rightarrow [\sqrt{n}]$).
\end{example}

\begin{example}
$\{(-1)^n\}_{n \in \N}$ rappresenta la successione $\{-1, 1, -1, 1, \dots \}$.
\end{example}

\begin{definition}
Una successione $\{x_n\}_{n \in \N}$ è crescente se $x_{n+1} > x_n$ per ogni $n$.
\end{definition}

\begin{definition}
Una successione $\{x_n\}_{n \in \N}$ è decrescente se $x_{n+1} < x_n$ per ogni $n$.
\end{definition}

\begin{definition}
Una successione $\{x_n\}_{n \in \N}$ è non crescente se $x_{n+1} \le x_n$ per ogni $n$.
\end{definition}

\begin{definition}
Una successione $\{x_n\}_{n \in \N}$ è non decrescente se $x_{n+1} \ge x_n$ per ogni $n$.
\end{definition}

Se una successione soddisfa una qualsiasi delle precedenti condizioni, allora essa si dice monotona.

\begin{example}
% TODO Usa reference anziché valori fissi
Le tre successioni mostrate in precedenza (3.12, 3.13, 3.14) sono rispettivamente crescente, non decrescente e non monotona.
\end{example}

\begin{definition}
Si dice che $L \in \R$ è il limite di $\{ x_n \}$ se per ogni intorno $B_r(L)$ di $L$ esiste $N \in \N$ tale che $x_n \in B_r(L)$ per ogni $n > N$.

Analogamente per ogni $r > 0$ esiste $N \in \N$ tale che $L-r < x_n < L+r$ per ogni $n > N$.

Si scrive
\begin{equation*}
\lim_{n \to +\infty}{x_n} = L
\end{equation*}
\end{definition}

\begin{example}
La successione $\{\frac{1}{n}\}_{n \in \N}$ ha limite 0.

Dobbiamo dimostrare che per ogni $r > 0$ esiste $N$ tale che se $n > N$ allora $\frac{1}{n} \in B_r(0)$.

Ovvero $|\frac{1}{n}| < r$, cioè $1 < n \cdot r$, quindi $n > \frac{1}{r}$.

Poniamo $N = [\frac{1}{r} + 1]$, allora $n > N$. $\implies n > \frac{1}{r} \implies x_n \in B_r(0)$. 

Quindi $\lim\limits_{n \to +\infty} \frac{1}{n} = 0$.
\end{example}