\chapter{Dodicesima lezione (13/11/2015)}

\section{Limite di successioni e limite di funzione}

Nella lezione precedente abbiamo introdotto il concetto di limite di funzione. Vediamo ora un teorema a riguardo del nesso che esiste tra il concetto di limite di successione e quello di limite di funzione.

\begin{theorem}[\bfseries Teorema di collegamento]
Sia $A$ un intervallo reale aperto, $x_0 \in A$ e $f: A \to \R$. Allora
\begin{equation*}
\lim_{x \to x_0} f(x) = L \iff (*)
\end{equation*}
$(*)$ esprime la seguente condizione: per ogni successione $\{x_n\}$ in $A\backslash\{x_0\}$ (ovvero $x_n \in A$ e $x_n \neq x_0 \; \forall n$) convergente a $x_0$ vale che
\begin{equation*}
\lim_{n \to +\infty} f(x_n) = L
\end{equation*}
\end{theorem}

Poiché il teorema indica ``se e solo se'', per dimostrarlo dobbiamo far vedere che vale l'implicazione in entrambi i sensi.

\begin{proof}
Dimostriamo la prima parte, ovvero che il limite $L$ implica $(*)$. Sia $
\lim_{x \to x_0} f(x) = L$.


Per definizione di limite, per ogni intorno $U = B_\epsilon (L)$ di $L$, esiste $\delta > 0$ tale che per ogni $x \in B'_\delta (x_0)$ vale $f(x) \in U$.

Sia $\{x_n\}$ una successione tale che $x_n \in A\backslash \{x_0\}$, allora $\lim_{n \to +\infty} x_n = x_0$. Devo dimostrare allora che $\lim_{n \to +\infty} f(x) = L$.

Per $\epsilon > 0$ dev'essere $f(x_n) \in B_\epsilon (L)$ definitivamente, che discende da $x_n \in B'_\delta (x_0)$ definitivamente.
\end{proof}

\begin{proof}
Dimostriamo ora il viceversa: supponiamo che valga $(*)$ e dimostriamo che allora $\lim_{x \to x_0} f(x) = L$.

Per assurdo, supponiamo che $L$ non sia il limite. Stiamo quindi dicendo che $\forall U$ intorno di $L$ esiste $\delta > 0$ tale che se $x \in B'_\delta (x_0) \implies f(x) \in U$.

Quindi esiste $U$ intorno di $L$ tale che per ogni $\delta > 0$ esiste $x_\delta \in B'_\delta (x_0)$ e tuttavia $f(x_\delta) \notin U$ (osserviamo che $x$ dipende da $\delta$).

Scegliamo $\delta = \frac{1}{n}$ e troviamo la successione $\{x_n\}$ tale che $x_n \in B'_{\frac{1}{n}} (x_0) \implies f (x_n) \notin U$. Ovvero $0 < |x_0 - x_n| < \frac{1}{n}$.

Per confronto di successioni, $\lim_{n \to +\infty} x_n = x_0$ con $x_n \neq x_0$ vale $\lim_{n \to +\infty} f(x_n) = L$ (per $(*)$).

Quindi $f(x_n) \in U$ definitivamente, che è assurdo.
\end{proof}

\section{Limite destro e sinistro}

\begin{example}
Consideriamo la funzione così definita e tracciamo il suo grafico:
\begin{equation*}
f(x) = \begin{cases}
\frac{x}{|x|} & x \neq 0 \\
0 & x = 0
\end{cases}
\end{equation*}

\begin{center}
\begin{tikzpicture}[scale=0.75]
\draw[->] (-3,0) -- (4.2,0) node[right] {$x$};
\draw[->] (0,-2) -- (0,2.5) node[above] {$y$};
\draw [smooth,domain=-2.75:0] plot({\x}, {-1});
\draw [smooth,domain=0.01:2.75] plot({\x}, {1});
\node at (0,0) {\tiny\textbullet};
\end{tikzpicture}
\end{center}

Osserviamo che $f(x)$ non ha limite per $x$ che tende a 0. Infatti, in ogni intorno di 0, $f(x)$ assume sia il valore 1 che il valore -1.
\end{example}

\begin{definition}
Sia $f: (x_0, b) \to \R$. Si dice che $L$ è il \emph{limite destro} di $f(x)$ in $x_0$ e si scrive
\begin{equation*}
L = \lim_{x \to x_0^+} f(x)
\end{equation*}
se per ogni $\epsilon > 0$ esiste $\delta > 0$ tale che
\begin{equation*}
x_0 < x < x_0 + \delta \implies f(x) \in B_\epsilon (L)
\end{equation*}
\end{definition}

\begin{definition}
Sia $f: (a, x_0) \to \R$. Si dice che $L$ è il \emph{limite sinistro} di $f(x)$ in $x_0$ e si scrive
\begin{equation*}
L = \lim_{x \to x_0^-} f(x)
\end{equation*}
se per ogni $\epsilon > 0$ esiste $\delta > 0$ tale che
\begin{equation*}
x_0 - \delta < x < x_0 \implies f(x) \in B_\epsilon (L)
\end{equation*}
\end{definition}

\begin{remark}
Se esiste il limite $L$ in $x_0$, $L$ è anche il limite destro e il limite sinistro.
\end{remark}
\begin{remark}
Se il limite destro e sinistro esistono entrambi e coincidono, allora esiste il limite.
\end{remark}

Infatti:
\begin{itemize}
\item dato il limite destro $\lim_{x \to x_0^+} f(x) = L$, fissato $\epsilon > 0$ esiste $\delta_1$ tale che $x_0 < x < x_0 + \delta_1 \implies f(x) \in B_\epsilon (L)$.
\item dato il limite sinistro $\lim_{x \to x_0^-} f(x) = L$, fissato $\epsilon > 0$ esiste $\delta_2$ tale che $x_0 - \delta_2 < x < x_0 \implies f(x) \in B_\epsilon (L)$.
\end{itemize}

Ciò che a noi serve è $\delta > 0$ tale che $x_0 - \delta < x < x_0 + \delta$ e $x \neq x_0 \implies f(x) \in B_\epsilon (L)$.

Se prendiamo $\delta = \min\{\delta_1, \delta_2\}$, allora entrambe le implicazioni sono soddisfatte e quindi vale la definizione di limite.

\section{Teorema di permanenza del segno}

\begin{theorem}[\bfseries Teorema di permanenza del segno]
Se 
\begin{equation*}
\lim_{x \to x_0} f(x) = L > 0
\end{equation*}
allora $\exists \delta > 0$ tale che $f(x) > 0$ per ogni $x \in B'_\delta (x_0)$.
\end{theorem}

\begin{proof}
Scegliamo $\epsilon = L$. Per definizione di limite $\exists \delta > 0$ tale che $x \in B'_\epsilon (x_0) \implies f(x) \in B_\epsilon (L)$. Possiamo quindi dire che:
\begin{gather*}
|f(x) - L| < \epsilon \\
|f(x) - L| < L \\
L - f(x) < L \\
f(x) > 0
\end{gather*}
\end{proof}

\section{Algebra dei limiti}

\begin{theorem}[\bfseries Algebra dei limiti]
Siano $f, g : A \to \R$ dove
\begin{equation*}
\lim_{x \to x_0} f(x) = L \qquad \text{e} \qquad \lim_{x \to x_0} f(x) = M
\end{equation*}
Allora:
\begin{enumerate}
\item $\lim\limits_{x \to x_0} (f(x) + g(x)) = L + M$
\item $\lim\limits_{x \to x_0} (f(x) \cdot g(x)) = L \cdot M$
\item se $f(x) = k$ costante allora $\lim\limits_{x \to x_0} f(x) = k$
\item se $\alpha \in \R$ allora $\lim\limits_{x \to x_0} \alpha f(x) = \alpha L$
\item se $y(x) \neq 0$ per $x \in A$ e $M \neq 0$ allora
$\lim\limits_{x \to x_0} \frac{f(x)}{g(x)} = \frac{L}{M}$
\item se $f(x) \le g(x)$ $\forall x$ allora $L \le M$
\item $\lim\limits_{x \to x_0} |f(x)| = |L|$
\end{enumerate}
\end{theorem}

Le dimostrazioni di queste implicazioni sono del tutto analoghe a quelle già enunciate per le successioni (teorema 4.13), grazie al teorema del collegamento (teorema 12.1). Dimostriamo comunque la prima implicazione a scopo didattico.

\begin{proof}
Sia $\{x_n\}$ una successione in $A\backslash\{x_0\}$ tale che
\begin{equation*}
\lim_{n \to +\infty} = x_0
\end{equation*}
Allora per il teorema del collegamento:
\begin{equation*}
\lim_{n \to +\infty} f(x_n) = L \qquad \text{e} \qquad \lim_{n \to +\infty} g(x_n) = M
\end{equation*}
Consideriamo
\begin{equation*}
\lim_{n \to +\infty} f(x_n) + g(x_n) = L + M
\end{equation*}
Quindi $\forall \{x_n\}$ in $A\backslash\{x_0\}$ che converge a $x_0$ si ha
\begin{equation*}
\lim_{n \to +\infty} (f(x_n) + g(x_n)) = L + M
\end{equation*}
Per il teorema del collegamento allora anche
\begin{equation*}
\lim_{x \to x_0} (f(x) + g(x)) = L + M
\end{equation*}
\end{proof}

\begin{example}
Calcoliamo a titolo di esempio un limite grazie all'algebra dei limiti.
\begin{equation*}
\lim_{x \to 2} \frac{x^2+2x}{1+x}
\end{equation*}
Possiamo applicare la quinta implicazione perché $1 + x \neq 0$ in un intorno di 2 (ad esempio $(-1, \infty) = A$). Inoltre:
\begin{equation*}
\lim_{x \to 2} (1+x) = \lim_{x \to 2} 1 + \lim_{x \to 2} x = 1 + 2 = 3
\end{equation*}
\begin{equation*}
\lim_{x \to 2} (x^2+2x) = \lim_{x \to 2} x^2 + \lim_{x \to 2} 2x = \lim_{x \to 2} x \cdot \lim_{x \to 2} x + \lim_{x \to 2} 2 \cdot \lim_{x \to 2} x = 2 \cdot 2 + 2 \cdot 2 = 8
\end{equation*}
In conclusione
\begin{equation*}
\lim_{x \to 2} \frac{x^2+2x}{1+x} = \frac{8}{3}
\end{equation*}
\end{example}