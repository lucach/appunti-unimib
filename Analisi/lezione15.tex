\chapter{Quindicesima lezione (01/12/2015)}

\section{Funzioni inverse (cont.)}

Al termine della lezione precedente abbiamo visto rapidamente un teorema sulla continuità delle funzioni inverse. Diamone orauna dimostrazione rigorosa.
\begin{theorem}
Data una funzione $f: [a, b] \to \R$ crescente e continua, allora $f^{-1}$ è continua.
\end{theorem}
\begin{proof}
L'immagine di $f$ è $[f(a), f(b)]$. Osserviamo che la funzione $f: [a, b] \to [f(a), f(b)]$ è suriettiva. Inoltre, essendo crescente è anche iniettiva.

Quindi la funzione è biunivoca e quindi esiste l'inversa $f^{-1}:[c,d] \to [a, b]$ (per semplicità $f(a)$ e $f(b)$ sono stati rinominati rispettivamente $c$ e $d$).

Anche $f^{-1}$ è crescente. Supponiamo per assurdo che non lo sia: in tal caso dev'essere che scelti
\begin{equation*}
c \le x < y \le d \implies f^{-1}(x) \ge f^{-1}(y)
\end{equation*}
Possiamo applicare $f$ a $f^{-1}$, senza invertire il verso della disuguaglianza, visto che la funzione $f$ è crescente. Quindi
\begin{equation*}
\implies f(f^{-1}(x)) \ge f(f^{-1}(y)) \implies x \ge y
\end{equation*}
Siamo arrivati all'assurdo, perché $x < y$ per ipotesi. Quindi anche $f^{-1}$ è crescente.

Dobbiamo ora dimostrare che è continua. Per semplicità, facciamolo prima considerando l'intervallo aperto $(c, d)$. 

Scegliamo $y_0 \in (c,d)$ e chiamiamo $x_0 = f^{-1} (y_0)$. Preso $\epsilon > 0$ deve esistere un intorno $U$ di $y_0$ tale che
\begin{equation*}
y \in U \implies f^{-1} (y) \in (x_0 - \epsilon, x_0 + \epsilon)
\end{equation*}
Sappiamo che $f(x_0 - \epsilon) < f(x_0 + \epsilon)$ perché la funzione è crescente.

Dato $y$ un punto nell'intorno ($f(x_0 - \epsilon) < y < f(x_0 + \epsilon)$), applichiamo $f^{-1}$ (che è crescente):
\begin{equation*}
x_0 - \epsilon < f^{-1} (y) < x_0 + \epsilon
\end{equation*}

Se scelgo $U$ intorno di $y_0$ tale che $U \subset (f(x_0 - \epsilon), f(x_0 + \epsilon))$ si ha che
\begin{equation*}
f(x_0 - \epsilon) < f(x_0) = y_0 < f(x_0 + \epsilon)
\end{equation*}

Quindi abbiamo dimostrato la continuità in $y_0$.
\end{proof}

\begin{proof}
Resta da mostrare che non vale solo sull'intervallo aperto $(c, d)$, ma anche negli estremi. Supponiamo $y_0 = c$ ed esprimiamo la condizione di continuità in $y_0$. Per ogni $\epsilon > 0$ cerchiamo un intorno $U$ di $y_0 = c$ tale che $\forall y \in U$ dev'essere $f^{-1} (y) \in B_\epsilon (x_0)$.

Dev'essere per forza $f^{-1} (c) = a$ perché $c$ è il minimo e $a$ è un punto di minimo (infatti $f$ è crescente).

Se $c \le y < f(a + \epsilon)$ allora:
\begin{gather*}
f^{-1} (c) \le f^{-1} (y) \le a + \epsilon \\
a \le f^{-1} (y) \le a + \epsilon \\
\end{gather*}

Quindi possiamo prendere $U$ intorno di $c$ tale che $U \cap [c,d] \subseteq [c, f(a+\epsilon)]$.

Analogo ragionamento può essere fatto sull'altro estremo.
\end{proof}

\begin{example}
La funzione $f(x) = \log x$ è continua. Infatti $\exp : [a, b] \to \R$ tale che $x \to e^x$ è continua e crescente. Essendo la funzione logaritmo inversa della funzione esponenziale, grazie al teorema possiamo affermare che $\log : [e^a, e^b] \to [a, b]$ è continua.
\end{example}

\begin{example}
Consideriamo la funzione $f(x) = x^a$ con $a \in \R, x > 0$. Tramite semplici passaggi:
\begin{equation*}
x^a = e^{\log {x^a}} = e^{a \cdot \log x} = \exp \circ g
\end{equation*}
dove $g(x) = a \cdot \log x$.
Abbiamo scritto $x^a$ come composizione di funzioni continue, quindi è continua.
\end{example}

\section{Asintoti di una funzione}
\begin{definition}
La retta $y = mx + q$ è un \emph{asintoto} della funzione $f$ per $x \to +\infty$ se
\begin{equation*}
\lim_{x \to +\infty} f(x) - (mx + q) = 0
\end{equation*}
\begin{itemize}
\item Se $m \neq 0$ l'asintoto si dice \emph{obliquo}.
\item Se $m = 0$ l'asintoto si dice \emph{orizzontale}.
\end{itemize}
\end{definition}

\begin{definition}
La retta $x = x_0$ è un \emph{asintoto verticale} della funzione $f$ per $x \to x_{0^+}$ se
\begin{equation*}
\lim_{x \to x_{0^+}} f(x) = +\infty \qquad \text{oppure} \qquad \lim_{x \to x_{0^+}} f(x) = -\infty
\end{equation*}
\end{definition}

\begin{example}
Consideriamo la funzione $f(x) = \log x$.

\begin{center}
\begin{tikzpicture}[scale=0.5]
\draw[->] (-2.5,0) -- (3.2,0) node[right] {$x$};
\draw[->] (0,-2.5) -- (0,2.2) node[above] {$y$};
\draw [domain=0.1:3] plot (\x, ln{\x});
\end{tikzpicture}
\end{center}


Essa presenta un asintoto verticale per $x \to 0^+$, poiché $\lim_{x \to 0^+} \log x = -\infty$.

La funzione invece non ha asintoto per $x \to +\infty$ perché $\lim \frac{\log x}{x} = 0$ e $\lim (\log x - 0) = +\infty$.
\end{example}

\begin{example}
La funzione $f(x) = \frac{1}{x} \sin x$ ha asintoto per $x \to +\infty$.

\begin{center}
\begin{tikzpicture}[scale=0.5]
\draw[->] (-2.5,0) -- (11.2,0) node[right] {$x$};
\draw[->] (0,-2.5) -- (0,2.2) node[above] {$y$};
\draw [domain=0.1:10] plot (\x, sin{deg(\x)}/\x );
\end{tikzpicture}
\end{center}

Calcoliamo $m$ e $q$:
\begin{align*}
m &= \lim_{x \to +\infty} \frac{f(x)}{x} = \frac{\sin x}{x^2} = 0 \\
q &= \lim_{x \to +\infty} \frac{\sin x}{x} - 0 = 0
\end{align*}
\end{example}

\begin{example}
Studiamo un eventuale asintoto obliquo per la funzione $f(x) = \log x - \sqrt{x^2-4}$ (definita per $x>2$).

\begin{equation*}
m = \lim_{x \to +\infty} \frac{\log x - \sqrt{x^2-4}}{x} = \lim_{x \to +\infty} \frac{\log x}{x} - \sqrt{\frac{x^2-4}{x^2}} = -1
\end{equation*}
\begin{equation*}
q = \lim_{x \to +\infty} f(x) - (-x) = \log x - \sqrt{x^2-4} + x
\end{equation*}
Studiamo il limite escludendo per il momento il logaritmo:
\begin{equation*}
\lim_{x \to +\infty} x - \sqrt{x^2 - 4} = \lim_{x \to +\infty} x - \sqrt{x^2 - 4} \cdot \frac{x + \sqrt{x^2 - 4}}{x + \sqrt{x^2 - 4}} = \lim_{x \to +\infty} \frac{x^2 - x^2 - 4}{x + \sqrt{x^2 + 4}} = 0
\end{equation*}

Quindi $\lim_{x \to +\infty} \log x + 0 = +\infty$, quindi la funzione non ha asintoto.

\end{example}

\section{Derivate}
Sia $f: [a,b] \to \R$ e $x_0 \in (a,b)$. Sia $P = (x_0, f(x_0))$ un punto preso sulla funzione e $Q = (x_0 + h, f(x_0+h))$ un altro punto della funzione.

\begin{center}
\begin{tikzpicture}[scale=1]
\draw[->] (-1,0) -- (2.2,0) node[right] {$x$};
\draw[->] (0,-0.5) -- (0,2.2) node[above] {$y$};
\draw [smooth,domain=-0.5:2] plot({\x}, {cos(\x r)});
\node at (0.35,0.75) {\footnotesize $Q$};
\node at (0.8,0.5) {\footnotesize $P$};
\end{tikzpicture}
\end{center}

Scriviamo l'equazione della retta passante per $P$ e $Q$:
\begin{equation*}
y = f(x_0) + \frac{f(x_0+h)-f(x_0)}{x_0 + h - x_0} \cdot (x-x_0)
\end{equation*}

Il coefficiente angolare di questa retta è (semplificando)
\begin{equation*}
\frac{f(x_0+h)-f(x_0)}{h}
\end{equation*}
Esso è detto anche \emph{rapporto incrementale} di $f$ in $x_0$.

\begin{definition}
Sia $f: [a, b] \to \R$ e $x_0 \in (a, b)$. Chiamiamo \emph{derivata} di $f$ in $x_0$ il limite
\begin{equation*}
\lim_{h \to 0} \frac{f(x_0+h)-f(x_0)}{h}
\end{equation*}
purché tale limite esista finito. Si indica $f'(x_0)$.
\end{definition}

Il ``limite'' delle rette $PQ$, per $Q \to P$ è:
\begin{equation*}
y = f(x_0) + f'(x_0) \cdot (x-x_0)
\end{equation*}
Questa è l'espressione analitica della retta tangente al grafico di $f$ in $x_0$.

\begin{definition}
Si dice che $f$ è \emph{derivabile} nell'intervallo se esiste la derivata in ogni punto.
\end{definition}

\begin{example}
Consideriamo ad esempio la funzione $f(x) = ax + b$.

\begin{center}
\begin{tikzpicture}[scale=0.75]
\draw[->] (-1.5,0) -- (2.2,0) node[right] {$x$};
\draw[->] (0,-1) -- (0,2.2) node[above] {$y$};
\draw [smooth,domain=-1:2] plot({\x}, {\x + 0.5});
\end{tikzpicture}
\end{center}

Se quanto abbiamo espresso sopra ha un senso, la tangente in questo caso dev'essere evidentemente la retta stessa. Applichiamo la definizione.
\begin{equation*}
\lim_{h \to 0} \frac{f(x+h)-f(x)}{h} = \lim_{h \to 0} \frac{a(x+h) + b  - (ax+b)}{h} = \lim_{h \to 0} \frac{ah}{h} = a
\end{equation*}
Quindi $f'(x) = a$ per ogni $x$. L'equazione della retta tangente è
\begin{align*}
y &= f(x_0) + f'(x)(x-x_0) \\
&= a(x_0) + b + a(x-x_0) \\
&= ax + b
\end{align*}
Che è esattamente coincidente all'equazione della retta iniziale.
\end{example}

\begin{example}
Consideriamo il polinomio $f(x) = x^n$ con $n > 0$ e calcoliamo la sua derivata applicando la definizione.

\begin{align*}
&\lim_{h \to 0} \frac{1}{h} \cdot (f(x+h)-f(x)) \\
&\lim_{h \to 0} \frac{1}{h} \cdot ((x+h)^n - x^n) \\
&\lim_{h \to 0} \frac{1}{h} \cdot \left(\binom{n}{0} x^nh^0 + \binom{n}{1} x^{n-1}h^1 + \ldots - x^n \right) \\
&\lim_{h \to 0} n \cdot x^{n-1} + o(h) = n \cdot  x^{n-1}
\end{align*}

\end{example}

\begin{example}
Consideriamo la funzione $f(x) = x^{\frac{1}{3}}$. Essa non è derivabile in 0.

\begin{center}
\begin{tikzpicture}[scale=1]
\draw[->] (-1.5,0) -- (2.0,0) node[right] {$x$};
\draw[->] (0,-1) -- (0,2.0) node[above] {$y$};
\draw [smooth,domain=-1:1] plot({\x}, {\x^(1/3)});
\end{tikzpicture}
\end{center}

Infatti:
\begin{equation*}
\lim_{h \to 0} \frac{1}{h} \cdot (f(h)-f(0)) = \lim_{h \to 0} \frac{1}{h} \cdot h^{\frac{1}{3}} = \lim_{h \to 0} h^{-\frac{2}{3}} = +\infty
\end{equation*}
\end{example}

\begin{example}
Consideriamo una funzione così definita:
\begin{equation*}
f(x) = \begin{cases}
x \cdot \sin \frac{1}{x} & x \neq 0 \\
0 & x = 0
\end{cases}
\end{equation*}
Controlliamo se è derivabile in 0:
\begin{equation*}
\lim_{h \to 0} \frac{1}{h} (f(h)-f(0)) = \lim_{h \to 0} \left(h \cdot \sin \frac{1}{h} \right) = \lim_{h \to 0} \sin \frac{1}{h}
\end{equation*}
Il limite non esiste perché il seno oscilla. Quindi la funzione è continua in 0 ma non è derivabile.
\end{example}

\begin{example}
Consideriamo una funzione così definita:
\begin{equation*}
f(x) = \begin{cases}
x^2 \cdot \sin \frac{1}{x} & x \neq 0 \\
0 & x = 0
\end{cases}
\end{equation*}
Controlliamo se è derivabile in 0:
\begin{equation*}
\lim_{h \to 0} \frac{1}{h} (f(h)-f(0)) = \lim_{h \to 0} \left(h^2 \cdot \sin \frac{1}{h} \right) = \lim_{h \to 0} h \cdot \sin \frac{1}{h} = 0
\end{equation*}
La funzione quindi è derivabile in zero.
\end{example}

\begin{definition}
Se $f: (a, b) \to \R$ è derivabile, la derivata definisce una funzione $f': (a, b) \to \R$.
\end{definition}

Se $f'$ è a sua volta derivabile, la sua derivata si indica $f''$ ed è detta derivata seconda di $f$. 

Induttivamente, la derivata $n$-esima $f^{(n)}$ è la derivata di $f^{(n-1)}$ (posto $f^{(0)} = 0$).

\begin{example}
\begin{align*}
f(x) &= x^n \\
f'(x) &= n \cdot x^{n-1} \\
f''(x) &= n \cdot (n-1) \cdot x^{n-2} 
\end{align*}
E così via...
\end{example}

\section{Derivabilità e continuità}

\begin{proposition}
Se $f : (a, b) \to \R$ è derivabile in $x_0$, allora $f$ è continua in $x_0$.
\end{proposition}

\begin{proof}
Essendo la funzione derivabile possiamo scrivere
\begin{equation*}
f'(x_0) = \lim_{h \to 0} \frac{1}{h} \cdot ((f(x_0+h)-f(x_0))
\end{equation*}
Consideriamo solo il limite della seconda parte e moltiplichiamo e dividiamo per $h$:
\begin{equation*}
\lim_{h \to 0} h \cdot \lim_{h \to 0} \frac{f(x_0+h)-f(x_0)}{h} = 0 \cdot f'(x_0) = 0
\end{equation*}

Essendo la differenza tra $f(x_0+h) - f(x_0) = 0$ allora
\begin{equation*}
\lim_{h \to 0} f(x_0+h) = f(x_0)
\end{equation*}
che è esattamente la proprietà che caratterizza la continuità.
\end{proof}

\section{Massimi e e crescenza (decrescenza)}

\begin{definition}
La funzione $f: (a,b) \to \R$ ha un \emph{massimo relativo} in $x_0$ (e $x_0$ è un \emph{punto di massimo relativo}) se esiste un intorno $V \subset (a,b)$ tale che $f(x) \le f(x_0)$ $\forall x \in V$.
\end{definition}

La definizione speculare vale ovviamente per il concetto di \emph{minimo relativo}.

\begin{definition}
La funzione $f$ è \emph{crescente} in $x_0$ se esiste un intorno $\sigma > 0, B_\sigma (x_0) \subseteq (a,b)$ tale che:
\begin{itemize}
\item $f(x) < f(x_0)$ per $x_0 - \sigma < x < x_0$
\item $f(x) > f(x_0)$ per $x_0 < x < x_0 + \sigma$
\end{itemize}
\end{definition}

La definizione speculare vale ovviamente per il concetto di funzione \emph{decrescente}.

\begin{remark}
Se $f: (a,b) \to \R$ è crescente in ogni punto, allora $f$ è crescente.
\end{remark}

\begin{proof}
Infatti, supponiamo $f$ continua e crescente in ogni suo punto. Siano $x, y$ punti tali che $a < x < y < b$; dobbiamo dimostrare che $f(x) < f(y)$.

Consideriamo una funzione $g: [x, y] \to \R$. La funzione $g$ è una restrizione di $f$ tale che $g(z) = f(z)$. Quindi la funzione $g$ è ancora continua. Per il teorema di Weierstrass, $g$ ha un massimo e un minimo assoluti.

Sia $c$ il punto di minimo assoluto. Sappiamo che $f$ è crescente nel punto $c$, quindi $\exists \sigma$ tale che per $c - \sigma < z < c + \sigma$ vale $f(z) < f(c)$.

Questo è possibile solo se $c = x$ perché $c$ è un punto di minimo assoluto. Quindi $x$ è un punto di minimo assoluto.

Con ragionamento analogo si arriva a far vedere che $y$ è un punto di massimo assoluto per $g$. Quindi $f(y) > f(x)$ perché il massimo è naturalmente maggiore del minimo.
\end{proof}

\begin{theorem}
Sia $f: (a,b) \to \R$ derivabile in $x_0$. Allora:
\begin{itemize}
\item se $f'(x_0) > 0$ allora $f$ è crescente in $x_0$;
\item se $f'(x_0) < 0$ allora $f$ è decrescente in $x_0$;
\item se $x_0$ è un punto di massimo o minimo relativo allora $f'(x_0)=0$.
\end{itemize}
\end{theorem}

Dimostriamo il primo caso (il secondo è speculare al primo).
\begin{proof}
La derivata in $x_0$ è positiva; ricordiamo che questo significa che
\begin{equation*}
\lim_{h \to 0} \frac{f(x_0+h)-f(x_0)}{h} > 0
\end{equation*}
Per il teorema della permanenza del segno:
\begin{equation*}
\frac{f(x_0+h)-f(x_0)}{h} > 0 \qquad \text{per } h \in B_\sigma (0)
\end{equation*}
Ora:
\begin{itemize}
\item se $x_0 - \sigma < x < x_0$ allora $f(x) - f(x_0) < 0$
\item se $x_0 < x < x_0 + \sigma$ allora $f(x) - f(x_0) > 0$
\end{itemize}
Queste due condizioni rappresentano esattamente la definizione di funzione crescente nel punto $x_0$.
\end{proof}

\begin{proof}
Dimostriamo il terzo caso. Se $x_0$ è un punto di massimo (minimo) relativo, $f$ non può essere né crescente né decrescente in $x_0$.

Per quanto abbiamo appena dimostrato:
\begin{itemize}
\item se fosse $f'(x_0) > 0$ la funzione sarebbe crescente, assurdo;
\item se fosse $f'(x_0) < 0$ la funzione sarebbe decrescente, assurdo.
\end{itemize}

Quindi dev'essere per forza $f'(x_0) = 0$.
\end{proof}