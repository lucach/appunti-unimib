\chapter{Ottava lezione (30/10/2015)}

\section{Serie armonica generalizzata}

Introduciamo una notazione comoda. Se $\sum_{j=0}^\infty x_j$ converge a $s$, cioè la successione delle somme parziali $\{s_n\}$ converge a $s$ (ovvero $s_n = \sum_{j=0}^n x_j$), allora scriviamo:
\begin{equation*}
\sum_{j=0}^\infty x_j = s
\end{equation*}

Al termine della lezione precedente avevamo visto la condizione di Cauchy. Ribadiamo ancora una volta che non vale il viceversa, mostrando un esempio.

\begin{example}[\bfseries Serie armonica]
La serie
\begin{equation*}
\sum_{j=1}^\infty \frac{1}{j}
\end{equation*}
soddisfa $\lim \frac{1}{j} = 0$ ma non converge. Tale serie viene chiamata ``serie armonica'' e possiamo considerare anche quella generalizzata (con $\alpha$ reale):
\begin{equation*}
\sum_{j=1}^\infty \frac{1}{j^\alpha}
\end{equation*}
\end{example}

\begin{theorem}
La serie armonica diverge; la serie armonica generalizzata converge per $\alpha > 1$ e diverge per $\alpha < 1$.
\end{theorem}

\begin{proof}
La serie armonica è una serie a termini positivi, quindi o converge o diverge. Confrontiamo i primi termini una successione che chiamiamo $\{z_j\}$ costruita in modo che ogni termine sia minore di quello della serie armonica. Scriviamo nella prima riga i termini della serie armonica e nella seconda quelli di $\{z_j\}$:
\begin{equation*}
x_1 = \frac{1}{1} \quad x_2 = \frac{1}{2} \quad x_3 = \frac{1}{3} \quad x_4 = \frac{1}{4} \quad x_5 = \frac{1}{5} \quad \ldots 
\end{equation*}
\begin{equation*}
x_1 = \frac{1}{1} \quad x_2 = \frac{1}{2} \quad x_3 = \frac{1}{4} \quad x_4 = \frac{1}{4} \quad x_5 = \frac{1}{8} \quad \ldots 
\end{equation*}

Formalmente $z_j = \frac{1}{2^k}$ dove $2^{k-1} < j \le 2^k$. Analizziamo ora $\{s_n\}$, che definiamo come la successione delle somme parziali di $\{z_j\}$.

\begin{align*}
s_1 &= z_1 = 1 \\
s_2 &= 1 + \frac{1}{2} \\
s_4 &= 1 + \frac{1}{2} + 2 \cdot \left(\frac{1}{4}\right) \\
s_8 &= 1 + \frac{1}{2} + 2 \cdot \left(\frac{1}{4}\right) + 4 \cdot \left(\frac{1}{8}\right)\\
\end{align*}

Si vede abbastanza facilmente che
\begin{align*}
s_{2^k} &= s_{2^{k-1}} + 2^{k-1} \cdot \frac{1}{2^k} \\
s_{2^k} &= s_{2^{k-1}} + \frac{1}{2} \\
s_{2^k} &= 1 + \frac{k}{2}
\end{align*}

A questo punto possiamo calcolare 
\begin{equation*}
\lim_{k \to +\infty} s_{2^k} = +\infty
\end{equation*}
Quindi $\{s_n\}$ non è limitata e quindi non può convergere. Allora $\sum z_j$ non converge, quindi diverge (essendo a termini positivi).

Osservando che $z_j \le \frac{1}{j}$ per ogni $j$ e che $\sum_{j=1}^\infty z_j$ diverge; allora $\sum_{j=1}^\infty \frac{1}{j}$ diverge.

\end{proof}

Per $\alpha < 1$ il teorema ci dice che $\sum_{j=1}^\infty \frac{1}{j^\alpha}$ diverge. Infatti se $\alpha < 1$ allora $j^\alpha < j$ e quindi $\frac{1}{j^\alpha} > \frac{1}{j}$. Quindi posso confrontare questa serie con $\frac{1}{j}$. Per il criterio del confronto diverge anch'essa.

Non dimostreremo il caso $\alpha > 1$, ma ci limitiamo ad osservare che se $\alpha = 2$ allora la serie è $\sum_{j=1}^\infty \frac{1}{j^2}$; abbiamo già visto che converge per confronto con quella di Mengoli.

\section{Criterio del rapporto per le serie}

\begin{theorem}[\bfseries Criterio del rapporto per le serie]
Sia $\sum_{j=1}^\infty x_j$ una serie a termini positivi. Se esiste il limite
\begin{equation*}
L = \lim_{j \to +\infty} \frac{x_{j+1}}{x_j}
\end{equation*}
allora:
\begin{enumerate}
\item se $L < 1$ la serie è convergente
\item se $L > 1$ la serie è divergente
\end{enumerate}
\end{theorem}

Mostriamo alcuni esempi di applicazione del teorema.
\begin{example}
La serie
\begin{equation*}
\sum_{n=0}^\infty \frac{x^n}{n!}
\end{equation*}
converge per ogni $x$. Infatti, applicando il criterio del rapporto
\begin{equation*}
\frac{x^{n+1}}{(n+1)!} \cdot \frac{n!}{x^n} = \frac{x}{n+1}
\end{equation*}
e $\lim \frac{x}{n+1} = 0$. Poiché il limite è minore di 1, la serie è convergente.

È interessante inoltre notare che per $x = 1$ la serie $\sum_{n=0}^\infty \frac{1^n}{n!}$ vale esattamente $e$.
\end{example}

\begin{example}
Studiamo la serie
\begin{equation*}
\sum_{n=0}^\infty nx^n \qquad x \in (0,1)
\end{equation*}
applicando il criterio del rapporto. Quindi:
\begin{equation*}
\frac{(n+1)x^{n+1}}{nx^n} = x \cdot \frac{n+1}{n}
\end{equation*}

Il limite di $x \cdot \frac{n+1}{n}$ vale $x$ e quindi per $0 < x < 1$ la serie converge.
\end{example}

Rimarchiamo il concetto che se $L = \lim \frac{x_{n+1}}{x_n} = 1$ non possiamo concludere né che la serie converge né che diverge. Infatti:
\begin{example}
$\sum_{j=1}^\infty \frac{1}{j}$ diverge e il suo limite è 1:
\begin{equation*}
\lim_{j \to +\infty} \frac{\frac{1}{j+1}}{\frac{1}{j}} = \frac{j}{j+1} = 1
\end{equation*}
\end{example}
\begin{example}
$\sum_{j=1}^\infty \frac{1}{j^2}$ converge e il suo limite è 1:
\begin{equation*}
\lim_{j \to +\infty} \frac{\frac{1}{(j+1)^2}}{\frac{1}{j^2}} = \frac{j^2}{(j+1)^2} = 1
\end{equation*}
\end{example}

Dimostriamo il teorema.
\begin{proof} \hfill
\begin{enumerate}
\item 
Se $L < 1$ scelgo $h$ in modo che $L < h < 1$. Allora so che $\frac{x_{j+1}}{x_j}$ è definitivamente minore di $h$.

Quindi esiste $N$ tale che $\frac{x_{j+1}}{x_j} < h$ per ogni $j > N$. Sviluppando i termini successivi:
\begin{align*}
x_{N+2} &< hx_{N+1} \\
x_{N+3} &< hx_{N+2} < h^2x_{N+1} \\
x_{N+K+1} &< h^kx_{N+1}
\end{align*} 

Quindi $\sum_{j=N+1}^\infty x_j$ converge perché $\sum h^kx_{N+1}$ converge (quest'ultima converge perché è una serie geometrica di ragione $h < 1$). Se converge a partire da $N+1$ in poi, allora tutta la serie converge. Quindi $\sum_{j=0}^\infty x_j$ converge.

\item
Se $L > 1$ procediamo in modo analogo al caso precedente. Prendiamo $1 < h < L$, quindi $\frac{x_{j+1}}{x_j} > h$ definitivamente. Come prima, osservando i passaggi, arriviamo a dire che $x_{N+k+1} > h^kx_{N+1}$.

Osserviamo che $\sum_{k=0}^\infty h^kx_{N+1} = x_{N+1}\sum_{k=0}^\infty h^k$ che diverge perché è una serie geometrica di ragione maggiore di 1. Quindi per confronto anche $\sum_{j=N+1}^\infty x_j$ diverge e quindi diverge anche $\sum_{j=0}^\infty x_j$.
\end{enumerate}
\end{proof}

\section{Criterio della radice}
\begin{theorem}[\bfseries Criterio della radice]
Sia $\sum_{j=1}^\infty x_j$ una serie a termini positivi tale che il limite
\begin{equation*}
L = \lim_{j \to +\infty} \sqrt[j]{x}
\end{equation*}
esiste ed è finito. Allora:
\begin{enumerate}
\item se $L < 1$ la serie converge
\item se $L > 1$ la serie diverge
\end{enumerate}
\end{theorem}

Al solito mostriamo alcuni esempi prima della dimostrazione.
\begin{example}
\begin{equation*}
\sum_{j=1}^\infty \left(\frac{1}{j!}\right)^j = \sum x_j
\end{equation*}
Applicando il criterio della radice:
\begin{equation*}
\sqrt[j]{x_j} = \sqrt[j]{\left(\frac{1}{j!}\right)^j} = \frac{1}{j!}
\end{equation*}
Quindi:
\begin{equation*}
\lim \sqrt[j]{x_j} = \lim \frac{1}{j!} = 0
\end{equation*}
Poiché 0 è minore di 1, la serie converge.
\end{example}

\begin{example}
\begin{equation*}
\sum_{j=2}^\infty \left(\frac{1}{\log j}\right)^j = \sum x_j
\end{equation*}
Applicando il criterio della radice:
\begin{equation*}
\sqrt[j]{x_j} = \sqrt[j]{\left(\frac{1}{\log j}\right)^j} = \frac{1}{\log j}
\end{equation*}
Quindi:
\begin{equation*}
\lim \sqrt[j]{x_j} = \lim \frac{1}{\log j} = 0
\end{equation*}
Poiché 0 è minore di 1, la serie converge.
\end{example}

Non ha a che fare con il criterio della radice, ma osserviamo che l'esempio appena fatto privo dell'elevamento a $j$ (quindi $\sum_{j=2}^\infty \frac{1}{\log j}$) diverge perché $\frac{1}{\log j}$ lo posso confrontare con $\frac{1}{j}$.

Poiché $\log j = o(j)$, allora $\frac{1}{\log j} > \frac{1}{j}$ definitivamente. Poiché $\sum \frac{1}{j}$ diverge, allora anche $\frac{1}{\log j}$ diverge.

\begin{proof}
Studiamo le due affermazioni:
\begin{enumerate}
\item 
Se $L < 1$, sia $L < h < 1$. Allora so che $\sqrt[j]{x_j} < h$ definitivamente. Elevando alla $j$ entrambi i membri si trova che $x_j < h^j$ definitivamente.

Essendo $h < 1$ allora $\sum_{j=0}^\infty h^j$ converge. Allora per confronto converge anche $\sum_{j=0}^\infty x_j$.

\item Se $L > 1$ allora posso dire direttamente che  $\sqrt[j]{x_j} > 1$ definitivamente. Elevando alla $j$ entrambi i membri si trova che $x_j > 1$ definitivamente.

Osserviamo che $\sum x_j > \sum 1$. Poiché $\sum 1$ diverge ($\lim 1 \neq 0$ e $1 + 1 + \ldots = +\infty$), allora anche $\sum x_j$ diverge.

Alternativamente possiamo dire che $\lim x_j$ non può essere 0 perché $x_j$ da un certo punto in poi è 1. Non è quindi soddisfatta la condizione di Cauchy e quindi la serie diverge.
\end{enumerate}
\end{proof}

\section{Confronto asintotico tra serie}
\begin{definition}
Date due serie a termini positivi $\sum_{j=0}^\infty x_j$ e $\sum_{j=0}^\infty y_j$, diciamo che sono \emph{asintoticamente equivalenti} se $x_j \sim y_j$, cioè $\lim \frac{x_j}{y_j} = 1$.
\end{definition}

\begin{theorem}[\bfseries Criterio del confronto asintotico]
Due serie a termini positivi asintoticamente equivalenti possono essere o entrambe convergenti o entrambe divergenti.
\end{theorem}

\begin{example}
Studiamo la serie 
\begin{equation*}
\sum_{n=1}^\infty \frac{n^2 + 3\sqrt{n}-4}{2n^3\sqrt{n+1}}
\end{equation*}

Osserviamo che il numeratore è asintotico a $n^2$ e il denominatore a $2n^3\sqrt{n}$. Il termine generale della serie è quindi:
\begin{equation*}
\frac{n^2 + 3\sqrt{n}-4}{2n^3\sqrt{n+1}} \sim \frac{n^2}{2n^3\sqrt{n}} = \frac{1}{2n\sqrt{n}} = \frac{1}{2n^{\frac{3}{2}}}
\end{equation*}

La serie di partenza è quindi asintoticamente equivalente a $\sum_{n=1}^\infty \frac{1}{2n^{\frac{3}{2}}}$. Siccome questa è una serie armonica generalizzata con $\alpha = \frac{3}{2}$, che è maggiore di 1, essa converge. Quindi, per il teorema del confronto asintotico, anche la serie di partenza converge.
\end{example}

\begin{example}
Studiamo la serie 
\begin{equation*}
\sum_{n=1}^\infty \frac{\cos n^2 + \sqrt{n}}{n}
\end{equation*}

Osserviamo che $-1 \le \cos n^2 \le 1$, quindi $\cos n^2 + \sqrt{n} \sim \sqrt{n}$.

La serie di partenza è quindi asintoticamente equivalente a $\sum_{n=1}^\infty \frac{\sqrt{n}}{n} = \frac{1}{n^\frac{1}{2}}$. Poiché $\frac{1}{2}$ è minore di 1, la serie diverge.
\end{example}

\begin{proof}
Siano $\sum x_j$ e $\sum y_j$ due serie a termini positivi asintoticamente equivalenti. Sappiamo quindi che $\lim \frac{x_j}{y_j} = 1$.

Possiamo considerare un intorno di 1, ad esempio $(\frac{1}{2}, \frac{3}{2})$. Dev'essere $\frac{1}{2} < \frac{x_j}{y_j} < \frac{3}{2}$ definitivamente. Moltiplicando tutti i termini per $y_j$ otteniamo:
\begin{equation*}
\frac{1}{2} \cdot y_j < x_j < \frac{3}{2} \cdot y_j
\end{equation*}
Ora:
\begin{enumerate}
\item se $\sum y_j$ diverge, allora $\sum \frac{1}{2} y_j$ diverge. Quindi $\sum x_j$ diverge per confronto.
\item se $\sum y_j$ converge, allora $\sum \frac{3}{2} y_j$ converge. Quindi $\sum x_j$ converge per confronto.
\end{enumerate}
\end{proof}
